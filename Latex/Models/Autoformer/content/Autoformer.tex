\subsection{Autoformer}

Autoformer là một mô hình được thiết kế để dự báo chuỗi thời gian dài hạn, bao gồm ba thành phần chính: Khối phân rã chuỗi (Series Decomposition Block), Cơ chế tự tương quan (Auto-Correlation Mechanism), và Bộ mã hóa/giải mã (Encoder/Decoder).

\subsubsection{Khối Phân Rã Chuỗi}

Khối phân rã chuỗi được sử dụng để tách chuỗi thời gian thành các thành phần xu hướng và mùa vụ. Phương pháp này sử dụng trung bình động để làm mượt các dao động định kỳ và làm nổi bật xu hướng dài hạn. Công thức cơ bản như sau:
\[
X_t = \text{AvgPool}(\text{Padding}(X))
\]
\[
X_s = X - X_t
\]
trong đó, $X_s$ và $X_t$ lần lượt là các phần mùa vụ và xu hướng của chuỗi.

\subsubsection{Cơ Chế Tự Tương Quan}

Cơ chế tự tương quan được thiết kế để phát hiện các phụ thuộc dựa trên chu kỳ và tập hợp các chuỗi con tương tự từ các chu kỳ ngầm. Cơ chế này thay thế cho self-attention trong Transformer và có độ phức tạp $O(L \log L)$, trong đó $L$ là chiều dài chuỗi.

\paragraph{Công Thức Tự Tương Quan}

Cơ chế tự tương quan bao gồm hai bước chính: tính toán tự tương quan và tổng hợp các giá trị tương quan.

\[
\text{ACF}(X, \tau) = \frac{1}{L} \sum_{t=1}^{L-\tau} X_t \cdot X_{t+\tau}
\]
trong đó, $\text{ACF}(X, \tau)$ là hàm tự tương quan, $X_t$ là giá trị tại thời điểm $t$, và $\tau$ là độ trễ.

Sau khi tính toán hàm tự tương quan, các giá trị được tổng hợp lại để tìm các chuỗi con tương tự, dựa trên công thức:
\[
Y_{t+\tau} = \sum_{k=1}^{K} \alpha_k X_{t+\tau_k}
\]
trong đó, $\alpha_k$ là trọng số tự tương quan, và $K$ là số lượng chuỗi con được chọn.

\subsubsection{Bộ Mã Hóa và Bộ Giải Mã}

Bộ mã hóa và bộ giải mã của Autoformer bao gồm các khối phân rã chuỗi và cơ chế tự tương quan. Bộ mã hóa xử lý dữ liệu đầu vào từ chuỗi thời gian quá khứ, trong khi bộ giải mã tinh chỉnh các thành phần mùa vụ và xu hướng để đưa ra dự báo.

\paragraph{Bộ mã hóa (Encoder)}

Bộ mã hóa bao gồm các khối phân rã và cơ chế tự tương quan để xử lý và trích xuất thông tin từ chuỗi đầu vào.

\[
H^{(l)} = \text{Decompose}(X^{(l)})
\]
\[
Z^{(l)} = \text{AutoCorrelation}(H^{(l)})
\]
trong đó, $H^{(l)}$ là đầu ra của lớp phân rã thứ $l$, và $Z^{(l)}$ là đầu ra của lớp tự tương quan thứ $l$.

\paragraph{Bộ giải mã (Decoder)}

Bộ giải mã nhận các thành phần xu hướng và mùa vụ từ bộ mã hóa và tiếp tục tinh chỉnh, sử dụng các khối phân rã và cơ chế tự tương quan để dự báo chuỗi thời gian trong tương lai.

\[
\hat{Y}^{(l)} = \text{Decompose}(Z^{(l)})
\]
\[
\hat{X}^{(l)} = \text{AutoCorrelation}(\hat{Y}^{(l)})
\]

\subsubsection{Quá Trình Hoạt Động}

Autoformer hoạt động theo quy trình sau:

\begin{enumerate}
    \item \textbf{Đầu vào và Phân Rã Chuỗi:} Chuỗi thời gian quá khứ $X_{en}$ được đưa vào bộ mã hóa. Bộ mã hóa phân rã chuỗi này thành các thành phần xu hướng và mùa vụ.
    \item \textbf{Bộ mã hóa (Encoder):} Áp dụng các khối phân rã và cơ chế tự tương quan để xử lý và trích xuất các thông tin cần thiết từ chuỗi đầu vào.
    \item \textbf{Bộ giải mã (Decoder):} Nhận các thành phần xu hướng và mùa vụ từ bộ mã hóa và tiếp tục tinh chỉnh. Sử dụng các khối phân rã và cơ chế tự tương quan để dự báo chuỗi thời gian trong tương lai.
\end{enumerate}