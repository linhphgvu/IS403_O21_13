\subsection{Random Forest}

\textbf{Định nghĩa}

Random Forest là phương pháp xây dựng một tập hợp cây quyết định cho các bài toán phân lớp và hồi quy. Cây quyết định thường có độ lệch thấp và phương sai cao, và Random Forest cải thiện hiệu suất bằng cách lấy trung bình các cây.

Giả sử \((X_t, Y_t)_{t \in \mathbb{Z}}\) là một dãy các biến ngẫu nhiên, trong đó:
\[ Y_t = f(X_t) + \epsilon_t \]
với \( \mathbb{E}[\epsilon_t \mid X_t] = 0 \). Mục tiêu là ước lượng hàm hồi quy 
\[ f(x) = \mathbb{E}[Y_t \mid X_t = x] \]
\textbf{Block Bootstrap cho Chuỗi Thời Gian:}
\begin{itemize}
    \item \textbf{Moving Block Bootstrap:} Chia dữ liệu thành các khối chồng lấn kích thước cố định, lấy mẫu lại để tạo chuỗi mới.
    \item \textbf{Circular Block Bootstrap:} Bao bọc chuỗi thời gian theo vòng tròn, đảm bảo điểm dữ liệu đều đại diện.
\end{itemize}
\textbf{Thuật Toán:}
\begin{itemize}
    \item \textbf{Đầu Vào:} Dữ liệu huấn luyện \( \{ (X_1, Y_1), \ldots, (X_n, Y_n) \} \), các tham số \( M, \alpha_n, m_{try}, \tau_n, l_n \).
    \item \textbf{Xây Dựng Cây cho \( j = 1 \) đến \( M \):}
    \begin{itemize}
        \item Lấy \( \alpha_n \) quan sát bằng block bootstrap với tham số \( l_n \).
        \item Chia nút chọn chia cắt tốt nhất từ tập con ngẫu nhiên của \( m_{try} \) đặc trưng.
    \end{itemize}
    \item \textbf{Đầu Ra:} Dự đoán cho một quan sát mới \( x \) là trung bình của dự đoán từ \( M \) cây:
    \[ \hat{f}(x) = \frac{1}{M} \sum_{j=1}^{M} \hat{f}_j(x) \]
\end{itemize}
