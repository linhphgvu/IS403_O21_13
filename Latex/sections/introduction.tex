\section{Giới thiệu}
Trong những thập kỷ gần đây, sự chú ý toàn cầu ngày càng tập trung vào suy thoái môi trường, đặc biệt là ô nhiễm không khí, do tác động sâu rộng của nó đối với sức khỏe con người, hệ sinh thái và sự ổn định của khí hậu. Hạt phấn mịn (PM2.5 và PM10) cùng với các khí như ozone (O3), dioxide nitơ (NO2), dioxide lưu huỳnh (SO2), và carbon monoxide (CO) đóng vai trò quan trọng trong sự suy giảm chất lượng không khí.
Trong bối cảnh này, nghiên cứu của Nhóm nhằm khám phá động lực chất lượng không khí trong môi trường đô thị, đặc biệt là ở ba thành phố lớn của Việt Nam: Hà Nội, Hạ Long và Việt Trì. Mỗi thành phố đại diện cho một ngữ cảnh kinh tế-xã hội và địa lý khác biệt, mang lại cái nhìn sâu sắc về các thách thức và cơ hội của quản lý ô nhiễm không khí.
Mục tiêu tổng thể là sử dụng các kỹ thuật phân tích chuỗi thời gian tiên tiến để dự báo và phân tích các chỉ số về chất lượng không khí. Nhóm đã sử dụng một loạt các thuật toán, từ các phương pháp truyền thống như Hồi quy Tuyến tính và Mô hình Trung bình Chuyển động Tích hợp Tự động (ARIMA) đến các kiến trúc học sâu tiên tiến như Mạng Nơ-ron Hồi quy (RNN), Đơn vị Hồi quy Cổng (GRU), Bộ nhớ Ngắn hạn - Dài hạn (LSTM), Hồi quy Vector (VAR), Rừng Ngẫu nhiên, Mô hình Mã hóa Dày Chuỗi Thời gian (TiDE), Autoformer, và Mạng Nơ-ron Đa Tầng (MLP). Bằng cách tận dụng những kỹ thuật này nhằm mục đích khám phá các mẫu, xu hướng và dự báo dữ liệu chất lượng không khí trong tương lai.
Các tập dữ liệu sử dụng bao gồm các số liệu đo thời gian thực về các chỉ số chất lượng không khí chính, bao gồm PM2.5, PM10, O3, NO2, SO2 và CO, được thu thập trong một khoảng thời gian dài. Những tập dữ liệu này cho phép khám phá các biến thể thời gian, sự đa dạng không gian và tương tác phức tạp giữa các chất gây ô nhiễm.
Các kết quả của nhóm đóng góp vào sự hiểu biết rộng lớn hơn về việc giải quyết các thách thức đa mặt do ô nhiễm không khí gây ra, thúc đẩy môi trường đô thị khỏe mạnh và bền vững hơn.
Ngoài ra, điều quan trọng là nhấn mạnh về vấn đề ô nhiễm không khí cấp bách mà Việt Nam đang phải đối mặt, đặc biệt là ở các thành phố lớn như Hà Nội, nơi chất lượng không khí đã nằm trong số tồi tệ nhất ở Đông Nam Á. Các yếu tố như mức độ cao của PM2.5 và các chất gây ô nhiễm khác đã gây ra những lo ngại sức khỏe đáng kể và tác động tiêu cực đến GDP của đất nước.