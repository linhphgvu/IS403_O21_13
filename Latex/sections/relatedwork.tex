\section{Nghiên cứu liên quan}

Trong những năm gần đây, đã có một sự gia tăng đáng kể trong nghiên cứu nhằm dự báo chất lượng không khí bằng cách sử dụng một loạt các kỹ thuật máy học, học sâu và thống kê.

Evgeniy Marinov, Dessislava Petrova-Antonova và Simeon Malinov, trong một nghiên cứu \cite{b1}, tập trung vào việc cải thiện độ chính xác của việc dự báo chất lượng không khí thông qua việc áp dụng phương pháp ARIMA. Nghiên cứu của họ, dựa trên dữ liệu từ các trạm giám sát chất lượng không khí tại Thành phố Sofia, Bulgaria, từ ngày 1 tháng 1 năm 2015 đến ngày 31 tháng 12 năm 2019, đã cho thấy hiệu quả của các mô hình ARIMA trong việc dự báo nồng độ ô nhiễm như CO, NO2, O3 và PM2.5.

S. H. Khaerun Nisa, Irfan Irfani và Utriweni Mukhaiyar, trong nghiên cứu của họ \cite{b2}, đã khám phá việc dự báo mức độ ô nhiễm không khí tại Jakarta bằng phương pháp phân tích Vector Autoregressive (VAR). Bằng cách phân tích dữ liệu chuỗi thời gian về chỉ số chất lượng không khí (AQI) và nồng độ PM2.5 thu thập từ ngày 16 tháng 8 đến ngày 25 tháng 9 năm 2023, để huấn luyện, và từ ngày 25 tháng 9 đến ngày 1 tháng 10 năm 2023, để kiểm tra, họ đã xác định mô hình VAR(2) tối ưu cho việc dự báo ô nhiễm không khí chính xác.

Khawaja Hassan Waseem và cộng sự đã khám phá tác động của các yếu tố khí hậu đối với nồng độ PM2.5 và phát triển mô hình dự báo trong một nghiên cứu gần đây \cite{b3}. Nghiên cứu của họ, kéo dài trong 30 ngày và 72 giờ cho các dự đoán hàng ngày và hàng giờ tương ứng, đã kết hợp dữ liệu chất lượng không khí, chất gây ô nhiễm và điều kiện khí hậu từ nhiều thành phố ở Pakistan. Sử dụng các mô hình máy học và học sâu bao gồm FbProphet và LSTM, họ đã phát hiện mô hình mã hóa-giải mã LSTM vượt trội so với các mô hình khác, đạt được cải thiện đáng kể về độ chính xác dự báo.

Hai tác giả, Marwa Winis Misbah Esager và Kamil Demirberk Ünlü \cite{b4}, đã áp dụng mô hình LSTM (Long Short-Term Memory) để dự báo nồng độ hàng giờ của hạt phấn mịn PM2.5 tại Tripoli, Libya. Họ đã sử dụng 100 epochs để huấn luyện mô hình, và kết quả tốt nhất đã được đạt được khi số nút được đặt là 20. Kết quả RMSE trên tập kiểm tra cho thấy mức độ lỗi thấp, khoảng 0.0146, chứng tỏ mô hình dự báo có độ chính xác khá cao.

Ngoài ra, các nghiên cứu gần đây cũng đã khám phá việc sử dụng các kỹ thuật mô hình hóa tiên tiến để cải thiện dự báo chất lượng không khí. Ví dụ, Abhimanyu Das và các đồng nghiệp đã giới thiệu TiDE (Time-series Dense Encoder) trong công việc của họ \cite{b5}. TiDE, một mô hình mới được tùy chỉnh cho dự báo chuỗi thời gian dài hạn, cho thấy khả năng hứa hẹn trong việc xử lý các biến đổi phi tuyến tính trong dữ liệu chuỗi thời gian. Các đánh giá thực nghiệm của TiDE đã cho thấy sự vượt trội hoặc tương đương với các phương pháp hiện tại trên các chỉ số dự báo dài hạn phổ biến trong thế giới thực, đồng thời tự hào về tốc độ suy luận và huấn luyện nhanh hơn đáng kể.

Ngược lại, nghiên cứu trước đó của Ong và đồng nghiệp \cite{b6} đã khám phá việc sử dụng các phương pháp dựa trên RNN để dự báo mức độ chất lượng không khí, trình bày một kỹ thuật động để tiền huấn luyện mô hình tập trung vào dự báo chuỗi thời gian nhiều bước trước. Tương tự, Lim và đồng nghiệp \cite{b7} đã sử dụng RNN để dự báo các chất gây ô nhiễm không khí khác nhau nhưng không tìm thấy sự khác biệt hoặc cải thiện đáng kể so với các mô hình truyền thống.

Bốn tác giả, Haixu Wu, Jiehui Xu, Jianmin Wang và Mingsheng Long, đã phát triển một mô hình gọi là Autoformer \cite{b8}. Mô hình này được thiết kế để giải quyết vấn đề dự báo chuỗi thời gian dài hạn, một thách thức đáng kể trong các ứng dụng thực tế như dự báo thời tiết cực đoan và lập kế hoạch tiêu thụ năng lượng dài hạn. Autoformer vượt trội so với các mô hình dựa trên Transformer trước đó bằng cách tích hợp các khối Phân rã và Mơ hình Tương quan Tự động dựa trên tính chu kỳ của chuỗi thời gian, cho phép mô hình khám phá và tổng hợp thông tin ở mức con-chuỗi. Kết quả thực nghiệm trên sáu thử nghiệm khác nhau, bao gồm năm ứng dụng thực tế từ năng lượng đến dịch bệnh, cho thấy rằng Autoformer đạt được độ chính xác hàng đầu với cải thiện tương đối 38\% so với các phương pháp hiện tại.

Những phương pháp và kết quả đa dạng này nhấn mạnh những nỗ lực tiếp tục để nâng cao các kỹ thuật dự báo chất lượng không khí, tận dụng cả các phương pháp thống kê truyền thống và các mô hình máy học tiên tiến.
